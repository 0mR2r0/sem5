%Paqueterias
\documentclass[12pt,letterpaper,final]{article}%Tipo de documento
\usepackage[utf8]{inputenc}
\usepackage[spanish,es-nodecimaldot,es-tabla]{babel}
\usepackage{amsmath}
\usepackage{amssymb}
\usepackage{graphicx}
\usepackage[left=2cm,right=2cm,top=2cm,bottom=2cm]{geometry}
\usepackage{titlesec}
\usepackage{hyperref}
\usepackage{pdflscape}
\usepackage{gensymb}
\usepackage{stix2}
\usepackage{siunitx}
\usepackage{listings}
\usepackage{tikz}
\usepackage[dvipsnames]{xcolor}
%%%%%%%%%%%%%%%%%%%%%%%%%%%%%%%%%%%%%%%%%%%%%%%%%%%%%%%%%%%%%
\lstset{language=Python,upquote=true,
	basicstyle=\ttfamily\small,numbers=left,
	numberstyle=\tiny,stepnumber=1,numbersep=5pt,
	backgroundcolor=\color{white},frame=single,tabsize=2,
	showspaces=false,showstringspaces=false,showtabs=false,
	breaklines=true,breakatwhitespace=true,escapeinside=||,
	keywordstyle=\color{blue!70},stringstyle=\color{green!70!black!70},
	commentstyle=\color{black!80}\it
}
%%%%%%%%%%%%%%%%%%%%%%%%%%%%%%%%%%%%%%%%%%%%%%%%%%%%%%%%%%%%%
\graphicspath{{images/}}
\baselinestretch
\renewcommand{\baselinestretch}{1.25}%Comando de interlineado
%%%%%%%%%%%%%%%%%%%%%
\usepackage{fancyhdr}%Encabezados
\pagestyle{fancy}
\fancyhf{}
\lhead[\leftmark]{Reporte semestral}
\rhead[]{\rightmark}
\lfoot[\thepage]{}
\rfoot[]{\thepage}
\renewcommand{\headrulewidth}{0.5pt}
\renewcommand{\footrulewidth}{0pt}
\fancypagestyle{plain}{
	\fancyhead[L]{Omar Arturo Castillo M\'endez}
	\fancyfoot[R]{\thepage}
	\renewcommand{\headrulewidth}{0.4pt}
	\renewcommand{\footrulewidth}{0.4pt}
}
\pagestyle{fancy}
%%%%%%%%%%%%%%%%%%%%%%%%%
%Datos del documento
\title{Reporte de final de primer semestre}
\date{}
\author{M.I.E. Omar Arturo Castillo M\'endez}
\begin{document}
	%%%%%%%%%%%%%%%%%%%%%%%%%%%%%%%%%%%%%%%%%%%%%%%%%%%%%
	\begin{titlepage}%Portada
		\centering
		\includegraphics[scale=1]{logo}
		\vspace{0.1cm}
		{\bfseries\LARGE Tecnol\'ogico Nacional de M\'exico \par}
		{\bfseries\LARGE Centro Nacional de Investigaci\'on y Desarrollo Tecnol\'ogico \par}
		\vspace{0.25cm}
		{\scshape\Large Departamento de Ciencias en Ingenier\'ia Electr\'onica \par}
		\vspace{0.25cm}
		{\scshape\Large Diseño, construcci\'on y puesta en marcha de un regenerador de energ\'ia para el desarrollo y la validaci\'on de estrategias de modelado matem\'atico \par}
		\vspace{0.25cm}
		{\itshape\Large Reporte de quinto semestre  \par}
		\vspace{0.25cm}
		{\bfseries Director de tesis: \par}
		\vspace{0.2cm}
		{Dr. V\'ictor Manuel Alvarado Mart\'inez \par}
		\vspace{0.2cm}
		{\bfseries Co-Directora de tesis: \par}
		\vspace{0.2cm}
		{Dra. Maria Guadalupe L\'opez L\'opez \par}
		\vspace{0.2cm}
		{\bfseries Comit\'e revisor: \par}
		\vspace{0.2cm}
		{Dr. Jos\'e Francisco G\'omez Aguilar \par}
		\vspace{0.2cm}
		{Dr. Ricardo Fabricio Escobar Jim\'enez  \par}
		\vspace{0.2cm}
		{Dr. Jarniel Garc\'ia Morales \par}
		\vspace{0.2cm}
		{\bfseries Presenta: \par}
		{M.I.E. Omar Arturo Castillo M\'endez \par}
		\vspace{0.2cm}
		{\today \par}
	\end{titlepage}
	%%%%%%%%%%%%%%%%%%%%%%%%%%%%%%%%%%%%%%%%%%%%%%%%%%%%%%%
	\tableofcontents{\thispagestyle{empty}}
	%\tableofcontents{\thispagestyle{empty}} Pagina sin numeracion
	\newpage
	%\listoffigures{\thispagestyle{empty}}
	%\newpage
	\setcounter{page}{1}

\section{Introduci\'on}
Este reporte presenta una revisión de algunos modelos de regeneradores de lecho empacado, un componente fundamental en diversos procesos industriales como la separación de gases, la purificación de líquidos y la recuperación de calor. Ademas, presentar una versión de un modelo de regenerador, así como la obtención de los parámetros que son necesarios para el desarrollo del mismo considerando aspectos físicos: las propiedades térmicas de la fase solida y gaseosa, del gas como el numero de Reynolds, porosidad del lecho, etc.

\section{Revisión de artículos}
\subsection*{Modelado de un regenerador-reactor de calor con propiedades de gas dependientes de la temperatura (Modeling a heat regenerator-reactor with temperature dependent gas properties)}
La principal característica de los regeneradores de energía es que, en el mismo espacio, puede contener dos gases de manera alternada. En el espacio vacío del regenerador, el gas más caliente fluye y cede su energía térmica a las partes sólidas (que deben tener una alta capacidad y densidad térmica). Posteriormente, un gas frío recupera ese calor durante un intervalo de tiempo. Para este trabajo, se consideró constante el coeficiente de transferencia de calor, pero se tomó como variable la compresibilidad del gas. Se modeló asumiendo un sistema adiabático, con la conductividad térmica del sólido infinitamente normal al flujo del gas y cero paralelo al flujo, con conexión ideal del gas, asumiendo que no existe dispersión del gas en las conexiones. El modelo se dividió en tres ecuaciones usando balances de energía y masa tradicionales. El sistema se consideró cerrado a la entrada, las variaciones de la velocidad al inicio del regenerador no afectan sus condiciones de entrada, y existe una pequeña acumulación de masa en el regenerador antes o después de completar el cambio de temperaturas. Durante el periodo de calentamiento, la velocidad del gas incrementa y la densidad disminuye (disminución neta de masa). El balance del gas en cualquier posición axial parte de la suma de razón de entalpía de entrada, la razón de salida y la transferida resulta en la razón de acumulación de la entalpía \cite{Kulkarni1992}.
\begin{equation}
	\frac{\partial T_g}{\partial t} = -\frac{1}{\epsilon L}[u\frac{\partial T_g}{\partial z}+ \frac{ha_p L}{\rho u C_p}(T_g-T_s)]
\end{equation}
Donde $u$ es la velocidad del gas en $\frac{m}{s}$, $\epsilon$ es la porosidad del regenerador
\newline
El balance de energía correspondiente al solido:
\begin{equation}
	\frac{dT_s}{dt} = \frac{ha_p}{(1-\epsilon)\rho_s C_{ps}}(T_g-T_s)
\end{equation}
El balance de masa:
\begin{equation}
	\frac{du}{dz} = -\frac{Rha_pL}{PM_wC_p}(T_g-T_s)
\end{equation}
En el balance de masa se esta considerando z como adimensional $z=\frac{x}{L}$
\newline
Las condiciones iniciales y de frontera:
\begin{equation*}
	T_g(0,z)=T_{g \,inicial} \quad T_g(t,0)=T_{g \,entrada} \quad T_s(0,z)= T_{s \, inicial} \quad u(t,0) = u_{entrada}
\end{equation*}
\subsubsection*{Parámetros del modelo}
La ley de los gases ideales:
\begin{equation}
	PV=N_m R T
\end{equation}
Donde: $N_m$ numero total de moles de cualquier gas, $R$ constante de los gases ideales y $T$ la temperatura del gas.
\begin{equation}
	a_p=\frac{6(1-\epsilon)}{d_p}
\end{equation}
Donde $a_p$ corresponde a la superficie especifica en el empaquetado, $d_p$ es el diámetro del relleno y $\epsilon$ a la porosidad del regenerador. 
\newline
La capacidad calorífica se puede obtener mediante la siguiente expresión, solo considerando que cambia conforme la variación de temperatura del gas\cite{green2018perry}, en caso contrario se puede considerar como constante:
\begin{equation}
	C_p=(0.79)[6.5+0.001(T_g)] + (0.21)(8.27+0.000258(T_g)-\frac{187700}{T_g^2}) \quad \frac{cal}{mol\cdot s \cdot K}
	\end{equation}
Las correlaciones semi-empiricas para el coeficiente de transferencia de calor pueden obtenerse mediante las siguientes expresiones\cite{Levenspiel1983}:
\begin{equation}
	Nu = 2 - 1.8 Re^{\frac{1}{2}}Pr^{\frac{1}{3}}
\end{equation}
\begin{equation*}
	Nu = \frac{hd_p}{k_g} \quad Re = \frac{d_p u\rho}{\mu} \quad Pr=\frac{C_p \mu }{k_g}
\end{equation*}

\subsection*{Solución mediante colocación triple para la operación periódica de regeneradores de calor (Solution by triple collocation for periodic operation of heat regenerators)}

El método basado en colocación triple ha sido desarrollado para la simulación de regeneradores de energía mediante un modelo lineal. El problema se reduce a un conjunto de ecuaciones algebraicas lineales. El problema del valor inicial en un regenerador al arranque y durante su operación puede ser resuelto por este método para cualquier modelo de ecuaciones.

Los regeneradores de calor se usan ampliamente en procesos industriales donde el gas está disponible a altas y bajas temperaturas. Algunos ejemplos importantes incluyen la recuperación de calor en centrales termoeléctricas, el precalentamiento de aire, la recuperación de calor de los gases de desperdicio en la fundición del hierro, en la manufactura del vidrio, el almacenamiento de energía solar, etc \cite{Ramachadran1984}.
\newline
\subsubsection*{Ecuaciones del modelo}
Las ecuaciones del modelo en su forma adimensional describen un empaquetado de un regenerador con la siguiente forma:
\newline
La primera ecuación representa un balance de energía, como es el cambio de temperatura entre la fase solida y el gas.
\begin{equation}
	\frac{1}{Pe}\frac{\partial ^2 \theta_h}{\partial x^2} - \frac{\partial \theta_h}{\partial x} - St(\theta_h -\theta_{phs}) = 0
\end{equation}
La siguiente corresponde a un balance de energía para indicar la temperatura de la partícula:
\begin{equation}
	\frac{1}{y^2}\frac{\partial}{\partial y}(y^2\frac{\partial \theta_{ph}}{\partial y}) = \frac{\partial \theta_{ph}}{\partial t} 
\end{equation}
La dirección en $x$ y $y$ se refiere a la posición axial adimensional dentro del regenerador y a la forma esférica de la partícula, respectivamente. $\theta_h$ es la temperatura de la masa del gas dentro del regenerador durante el periodo de calentamiento, para una posición en $x$ y un tiempo $t$. $\theta_{ph}$ es la temperatura de la partícula durante el periodo de calentamiento, para una posición $x$ dentro del regenerador y una posición radial en $y$ en el tiempo $t$. Finalmente, $\theta_{phs}=\theta_{ph}(x,y=1,t)$ es la temperatura externa de la superficie de la esfera.
\newline
Las condiciones de frontera son:
\begin{equation}
	x = 0, \qquad \frac{1}{Pe}\frac{\partial \theta_h}{\partial x} = \theta_h - \theta_{h,en}
\end{equation} 
\begin{equation}
	x = 1, \qquad \frac{\partial \theta_h}{\partial x} = 0
\end{equation}
\begin{equation}
	x = 0, \qquad \frac{\partial \theta_{ph}}{\partial y} = 0
\end{equation} 
\begin{equation}
	x = 1, \qquad \frac{1}{Bi}\frac{\partial \theta_{ph}}{\partial y} = \theta_h - \theta_{phs}
\end{equation}
Donde $\theta_{h,en}$ es la entrada adimensional de la temperatura del gas caliente. Las ecuaciones y condiciones de frontera están basadas en la suposición de que la transferencia de energía ocurre mediante flujo másico. La dispersión es axial en la fase gaseosa, la transferencia de energía entre el fluido, las partículas y su parte interna es por conducción. No se consideró la transferencia de calor por radiación y conducción entre partículas. Las propiedades físicas y los parámetros de transporte se asumen independientes de la temperatura. La condición inicial para la parte interna de las partículas es la siguiente:
\begin{equation}
	t=0, \qquad \theta_{ph} = \theta_{ph0}(y,x)
\end{equation}
El valor de $\theta_{ph0}$ representa la temperatura inicial que se distribuye para el periodo de enfriamiento y la temperatura de distribución al final del periodo de calentamiento, si solo se considera un solo ciclo el valor $\theta_{ph0}=0$.
\newline
\subsubsection*{Números adimensionales}
\begin{itemize}
	\item Número de Stanton [St] = $\frac{h a_p L}{u_g \rho_g C_{pg}}$
	\item Número de Biot [Bi] = $\frac{h R}{\lambda_e}$
	\item Número de Peclet [Pe] = $\frac{ u_g \rho_g C_{pg} L}{\lambda_{ax}}$
	\item Tiempo adimensional [t] = $\frac{t_a \lambda_e}{\rho_s C_{ps} R^2}$ o $ \frac{h a_p t_a }{(1-\epsilon_B)\rho_s C_{ps}} = 3 (Bi) t $
	
\end{itemize}

\subsection*{Modelos matemáticos para la simulación de regeneradores térmicos: Un análisis del estado del arte (Mathematical models for the simulation of thermal regenerators:
	A state-of-the-art review)}
Los regeneradores de lecho fijo pueden clasificarse en dos tipos: rotatorios y de lecho fijo. En un regenerador de lecho fijo, los materiales que almacenan el calor permanecen estáticos en su interior. El uso de válvulas solenoides permite el paso alternado de gas caliente y frío. Durante el periodo de calentamiento, el gas caliente fluye a través del lecho, transfiriendo su calor a los materiales. Posteriormente, un flujo de gas frío pasa a través del lecho, absorbiendo el calor almacenado durante el periodo de calentamiento. Esto corresponde al periodo de enfriamiento.
Es común el uso de materiales como aluminio, acero y vidrio para el armado del lecho, debido a sus propiedades térmicas y estructurales \cite{SADRAMELI2016}.
\newline
El modelo esta basado según las siguientes consideraciones:
\begin{itemize}
	\item Las propiedades térmicas y físicas del gas y el sólido son constantes e independientes de la temperatura y posición. \item No existe pérdida de calor del regenerador hacia el ambiente. \item No hay fuentes de energía que ocurran dentro del regenerador ni reacciones químicas internas. \item La capacidad térmica del gas dentro del lecho en cualquier instante es pequeña comparada con el lecho mismo. \item Los coeficientes de flujo másico y de transferencia de calor son constantes. \item La velocidad de entrada y la temperatura para cada fluido son uniformes a través de la sección transversal y constantes en el tiempo. \item La conductividad térmica del sólido es infinitamente mayor en la dirección normal al flujo del gas e infinitamente menor en la dirección paralela al flujo. \item No existe dispersión de flujo dentro del regenerador. \item La transferencia de calor del fluido es insignificante tanto longitudinal como transversalmente. \item El espacio vacío y la superficie de contacto del lecho son uniformes. \item El tiempo de residencia del gas dentro del lecho es insignificante en comparación con el periodo. \item La transferencia de calor por radiación es pequeña en comparación con otros mecanismos de transferencia.
\end{itemize}
\subsubsection*{Modelo}
El modelo para un regenerador de lecho fijo, para la fase solida:
\begin{equation}
	M_s C_{ps}\frac{\partial T_s}{\partial \theta} = h a (T - t_h)
\end{equation}
Para la fase del fluido:
\begin{equation}
	\frac{G C_{pg}}{u}\frac{\partial t_h}{\partial \theta} + G C_{pg}\frac{\partial t_h}{\partial x} = ha(T_s - t_g)
\end{equation}
Las condiciones iniciales y de equilibrio del proceso son:
\begin{equation*}
	t_h(0,t_h) = t_{h,i} = cte , \qquad 0<t_h<P_h
\end{equation*}

\begin{equation*}
	t_h(0,t_c) = t_{c,i} = cte , \qquad 0<t_c<P_c
\end{equation*}
Para las condiciones de equilibrio:
\begin{equation*}
	T_{s,h}(x,t_h=0) = T_{s,c}(x,t_c=P_c)
\end{equation*}
\begin{equation*}
	T_{s,h}(x,t_h=0) = T_{s,c}(x,t_c=P_c)
\end{equation*}
\begin{equation*}
	0<x<L
\end{equation*}
Se resolvió el modelo redefiniendo los parámetros de longitud, temperatura y tiempo de manera unidimensional:
\begin{equation*}
	y = \frac{x}{L}
\end{equation*}
\begin{equation*}
	z_1 = \frac{(t-\frac{x}{u_1})}{P_1}
\end{equation*}
\begin{equation*}
	z_2 = \frac{(t-\frac{x}{u_2})}{P_2}
\end{equation*}
\subsection*{Revisión del Diseño y Modelado de Intercambiadores de Calor Regenerativos (Review of Design and Modeling of Regenerative
	Heat Exchangers)}
Los regeneradores de energía son dispositivos que usan de manera indirecta el intercambio de energía con medios fríos y calientes. El uso principal se encuentra en el área de la metalurgia, el tratamiento y precalentamiento del aire, la recuperación del calor residual y en turbinas. En este trabajo se aplicó el método abierto de Willmott, que mostró una gran estabilidad y permitió la inclusión de ecuaciones que describen la transferencia de calor y la caída de presión. La ventaja de los regeneradores sobre los recuperadores es su mayor área de contacto en relación con su capacidad volumétrica. Varios materiales y formas pueden usarse como material de relleno en el regenerador, ya que los sólidos tienen una gran capacidad calorífica en comparación con los gases. Para pequeños regeneradores, pueden usarse rellenos como los de panal de abeja, esferas, monolitos, anillos, monturas, etc.  \cite{Kilkovsky2020}
\subsubsection*{Modelo}
\begin{equation}
	M_b C_{p,b}\frac{\partial T_b}{\partial t} = h_t A (T_g - T_b) 
\end{equation}
Donde $M_b$ es la masa del relleno ($kg$), $T_b$ es la temperatura media del relleno ($^\circ C$), $h_t$ es el coeficiente global de transferencia de calor $(\frac{W}{m^2 K})$, $C_{p,b}$ es la capacidad calorífica del material de relleno $(\frac{J}{kg K})$, $A$ es la superficie total de transferencia$(m^2)$.
\begin{equation}
	mg C_{p,g} L \frac{\partial T_g}{\partial y} + M_g C_{p,g} \frac{\partial T_g}{\partial y} = h_t A (T_b - T_g) 
\end{equation} 
Donde $m_g$ es el flujo másico del gas $(\frac{kg}{s})$, $C_{p,g}$ es la capacidad calorífica del gas y $L$ es la longitud del regenerador $(m)$. 
\subsubsection*{Parámetros del modelo}
\textbf{Espacio vacío}
\newline
El valor de $\epsilon$ es la razón del volumen disponible respecto a el volumen total del relleno:
\begin{equation}
	\epsilon = \frac{V_b -V_p}{V_b} \times 100 = \frac{V_m}{V_b} \times 100
\end{equation}
Donde $V_b$ es el volumen total del relleno ($m^3$), $V_p$ es el volumen del material de relleno ($m^3$) y $V_m$ es el espacio libre del relleno ($m^3$)
\newline
\textbf{Diámetro de partícula}
\newline
El diámetro puede definirse como el de una esfera.
\begin{equation}
	d_v=(\frac{6}{\pi}V_p)^{\frac{1}{3}}
\end{equation}
\textbf{Coeficiente global de transferencia de calor}
\newline
Para calcular este coeficiente se puede obtener de la siguiente manera segun Hausen\cite{Hausen1976}:
\begin{equation}
	h_t= \frac{1}{c} + \frac{1}{2(n+2)\lambda_b}\phi H + \frac{1}{h_r}
\end{equation}
El termino del coeficiente de fricción se obtuvo mediante el numero de Reynolds y usando la relacion de Hicks\cite{Hicks1970}.
\newline
La función $\phi H$ llamado factor de Hausen, intenta representar el efecto de la rapidez del cambio de temperatura dentro del empaquetado al inicio de un periodo de calentamiento o enfriamiento\cite{HINCHCLIFFE1981}.
\begin{equation}
	\phi H = 1 - \frac{d^2}{4 \alpha (n + 3)^2 - 1}  [\frac{1}{P^\prime} + \frac{1}{P^{\prime\prime}}]
\end{equation}
El coeficiente $\alpha$ es la difusividad térmica, $P$ es el periodo de calentamiento o enfriamiento $(s)$, $\epsilon =27$ para esferas.
\newline
Las consideraciones para el desarrollo del modelo fueron las siguientes:
\begin{itemize}
	\item El flujo másico es constante en ambos tiempos.
	\item La capacidad calorífica del gas dentro de los canales es lo suficientemente pequeña en comparación con la del solido, que puede despreciarse.
	\item Los coeficientes de transferencia y las propiedades térmicas del almacenamiento de calor y del gas no varían a lo largo de un período y son idénticos en todas las partes del regenerador.
	\item La conductividad en dirección longitudinal es despreciable. 
\end{itemize}
\textbf{Condiciones de frontera}
\begin{itemize}
	\item Las temperaturas en las entradas de los periodos de enfriamiento o calentamiento son constantes.
	\item La superficie a lo largo del regenerador y hasta el final de cada periodo es el mismo que al principio seguido del periodo anterior(calentamiento o enfriamiento). 
\end{itemize}
\begin{equation*}
	T_b^\prime (y,0) = T_b^{\prime \prime} (L - y,P^{\prime \prime})
\end{equation*}

\subsection*{Transferencia de calor en un lecho fijo para almacenamiento de energía térmica}
El modelo abordado por López\cite{Lopez2013}, consideró lo siguiente para desarrollo del modelo. Para el solido, capacidad de almacenamiento de calor constante, despreciando las variaciones de temperatura en la dirección radial, no existe generación interna de calor. Para el fluido, fluido newtoniano, no hay transferencia de masa, se desprecian los efectos de transferencia por radiación, no hay cambio de fase. 
\subsubsection*{Condiciones iniciales y de frontera}
La condición inicial esta en ambas ecuaciones(fluido y solido), el aire se encuentra a una temperatura inicial y uniforme $T_0$, 
\begin{equation}
	T=T_0 , \qquad t=0
\end{equation} 
Para la condición de frontera en la entrada del lecho:
\begin{equation}
	T=T_{en}
\end{equation}
En la salida del regenerador, tanto para el solido como para el fluido, al no exitir mas lecho, el solido no puede intercambiar mas calor y ademas el fluido en la salida no intercambia calor con el fluido al frente:
\begin{equation}
	\frac{\partial T}{\partial x} = 0, \qquad x=L
\end{equation}
La ecuación de conservación de energía para la fase solida del regenerador:
\begin{equation}
	\rho_s(1-\epsilon_a)C_{ps}\frac{\partial \theta}{\partial t} = (1-\epsilon_a)K_{s,x}\frac{\partial^2 \theta}{\partial^2 x^2} - ha_p(\theta - T)
\end{equation}
Para la fase del fluido que envuelve las partículas del fluido:
\begin{equation}
	\rho_a\epsilon_a C_{pa} \frac{\partial T}{\partial t} = \epsilon_a K_{a,x} \frac{\partial^2 T}{\partial^2 x^2} + ha_p(\theta - T) - \rho_a\epsilon_a C_{pa} u \frac{\partial T}{\partial x} + U a_w(T_0 - T )
\end{equation}
\subsection{Conclusión de la revisión del estado del arte}

Se observó que las ecuaciones que indican el comportamiento del regenerador son similares a las empleadas en otros procesos de transferencia de calor y masa \cite{Kilkovsky2020} \cite{Kulkarni1992} \cite{SADRAMELI2016}, lo que permite un análisis comparativo y una comprensión más profunda de la dinámica del sistema.

Los modelos que solo se basan en el balance de energía pueden proporcionar una aproximación útil, pero no capturan la complejidad del transporte de masa, lo que puede resultar en predicciones inexactas del comportamiento del regenerador.

Por otra parte, la inclusión del balance de masa en el modelo permite una descripción más completa del proceso, incluyendo la transferencia de masa entre las fases fluida y sólida \cite{Ramachadran1984}. De esta manera, se está revisando cómo es un modelo de regenerador que solo involucra balances de energía, debido a que para el cálculo de parámetros como el número de Reynolds, involucra de manera indirecta los efectos de la velocidad del gas dentro del regenerador. Posteriormente, se comparará con modelos que incluyan balances de masa, los cuales expresan de manera directa el efecto de la velocidad en el dispositivo.
%\\\\\\\\\\\\\\\\\\\\\\\\\\\\\\\\\\\\\\\\\\\\\\\
\newpage

\subsection{Propuesta de soluci\'on} 
Se propone dise\~nar una planta de pruebas a partir de consideraciones t\'ecnicas y del modelo t\'ermico y de din\'amica de fluidos. De tal manera, este sistema servir\'a como punto de partida para analizar otras t\'ecnicas de modelado y compararlas entre si. Se espera que sirva de referente para el estudio de otro tipo de lechos o empaques para trabajos a futuro.
\subsection{Objetivo general}
Diseñar, construir, y poner en operaci\'on una planta piloto de un regenerador de energ\'ia de lecho empacado, que sirva como estaci\'on de prueba para validar estructuras matem\'aticas que representen la din\'amica de la
planta y que sirvan para la soluci\'on de problemas de diseño, optimizaci\'on y control de estos sistemas.
\subsection{Objetivos espec\'ificos}
\begin{itemize}
	\item Obtener el modelo del comportamiento del intercambiador de calor mediante las ecuaciones gobernantes para una etapa de calentamiento o enfriamiento.
	\item Obtener un prototipo  de regenerador de energía de doble lecho empacado con un monolito metálico.
	\item Caracterizar el comportamiento del medio continuo y térmico del regenerador de energía para un periodo de calentamiento o enfriamiento del aire.
	\item Analizar la dinámica del sistema y caracterizar el comportamiento de un ciclo completo en régimen pseudo-estacionario.
	\item Modelar el comportamiento térmico de un ciclo completo por identificación de sistemas(técnicas lineales o no lineales).
	
\end{itemize}
\newpage
\section{Caso de estudio}
\subsection{Consideraciones de modelado }
Las suposiciones para el desarrollo del modelo se tomaron como primer exploración lo siguiente: La temperatura del gas solo para el proceso de calentamiento es constante, el flujo másico permanece constante, las propiedades térmicas del gas y del solido son constantes y no varían a lo largo del regenerador.
\subsection{Condiciones de iniciales y frontera}
La condición inicial muestra que en la fase solida, como en el aire se encuentran a una temperatura inicial y uniforme, suponiendo que es igual a la temperatura ambiente
\begin{equation}
	T_b=T_a , \qquad t=0
\end{equation} 
Para la condición de frontera en la entrada del lecho:
\begin{equation}
	T_{g,0}=T_{en}
\end{equation}
En la salida del regenerador, tanto para el solido como para el fluido, al no existir mas lecho, el solido no puede intercambiar mas calor y ademas el fluido en la salida no intercambia calor con el fluido al frente:
\begin{equation}\label{Solido}
	\frac{\partial T}{\partial x} = 0, \qquad x=L
\end{equation}
\subsection{Modelo}
El modelo de estudio se consideraron dos sistemas de ecuaciones:
\begin{equation}\label{Gas}
	M_b C_{p,b}\frac{\partial T_b}{\partial t} = h_t A (T_g - T_b) 
\end{equation}
 
\begin{equation}
	m_g C_{p,g} L \frac{\partial T_g}{\partial x} + M_g C_{p,g} \frac{\partial T_g}{\partial t} = h_t A (T_b - T_g) 
\end{equation}
\subsubsection{Desarrollo del modelo}
Para darle solución a las ecuaciones \ref{Solido} y \ref{Gas}, se dividió el problema solucionando el modelo sin los parámetros para ver las variables para ver el comportamiento de las ecuaciones, para que posteriormente incluirlos y ver el comportamiento completo del modelo.
Se optó por solucionar el modelo usando el método de diferencias finitas que a continuación se mostrará el desarrollo matemático.
\newline 
 \begin{equation}
 	\frac{\partial T_b}{\partial t} = (T_g - T_b) 
 \end{equation}
 
 \begin{equation}
 	\frac{\partial T_g}{\partial x} + \frac{\partial T_g}{\partial t} = (T_b - T_g) 
 \end{equation}
\subsubsection{Aplicación del método de diferencias finitas}
En este método de discretización, se caracteriza por convertir las ecuaciones diferenciales parciales, a un conjunto de ecuaciones por diferencias que sirven para aproximar a las ecuaciones originales. Se utilizó la aproximación de tipo central, porque es con el mecanismo de transferencia de calor, porque cada punto que se discretiza a lo largo del regenerador $0\leq x\leq L$ se va obteniendo el valor de la temperatura tomando en cuenta el valor del punto anterior y el valor del punto siguiente para obtener el punto que se esta evaluando.
\begin{equation*}
	\frac{d T_b}{d t} = T_{g_i} - T_{b_i}
\end{equation*} 
\begin{equation*}
	\frac{d T_g}{d t} = (T_{b_i} - T_{g_i}) - \frac{T_{g_{i+1}} - T_{g_{i-1}} }{2\Delta x}   
\end{equation*}
Al aplicar las diferencias finitas para indicar la discretización en el espacio se coloca con un subíndice $i$ en cada variable que dependan de $x$. De tal manera que las ecuaciones que inicialmente era dependiente tanto de $x$ y $t$ ahora solo dependen de $t$ y se indica en una derivada ordinaria.
\newline
Posteriormente como también se discretizó en el tiempo los términos del lado izquierdo de las ecuaciones queda de la siguiente manera.
\begin{equation*}
	\frac{T_{b_i}^{n + 1} - T_{b_i}^{n}  }{\Delta t} = T_{g_i}^n - T_{b_i}^n
\end{equation*} 
\begin{equation*}
	\frac{ T_{g_i}^{n + 1} - T_{g_i}^{n}   }{\Delta t} = (T_{b_i}^n - T_{g_i}^n) - \frac{T_{g_{i+1}}^n - T_{g_{i-1}}^n }{\Delta x}   
\end{equation*}
 Cuando se discretiza en el tiempo se debe tomar en cuenta el criterio de convergencia Courant-Friedrichs-Lewy (CFL) \cite{ZillCullen2008} $ \frac{u\Delta t}{\Delta x}$ para mantenerla estabilidad y la precisión de la solución numérica. Con la discretización en el tiempo lo que se busca encontrar los valores en el siguiente instante de tiempo, por eso la derivada en función del tiempo se utiliza su aproximación en diferencias hacia adelante solo para el tiempo.
 \newline
 De tal manera, se iguala al termino con la posición en el tiempo hacia adelante($n+1$), entonces el sistema de ecuaciones es:
 
\begin{equation*}
	T_{b_i}^{n + 1} =  T_{b_i}^{n} + \Delta t (T_{g_i}^n - T_{b_i}^n)
\end{equation*} 
\begin{equation*}
	T_{g_i}^{n + 1} = T_{g_i}^{n} + \Delta t(T_{b_i}^n - T_{g_i}^n) -   \frac{\Delta t}{\Delta x} ( T_{g_{i+1}}^n - T_{g_{i-1}}^n  )  
\end{equation*} 
  Con las ecuaciones anteriores se dio solución usando Python, del cual se obtuvieron las siguientes graficas, el código esta en el anexo \ref{Codigo1}:
  
  \begin{figure}[ht!]
  	\begin{center}
  		\includegraphics[scale=.6]{solucion.png}
  		\caption{Solución del sistema de ecuaciones}	
  	\end{center}	
  \end{figure}
  En el gráfico del lado izquierdo, se muestra el comportamiento de cómo sería la temperatura a lo largo del regenerador para un instante de tiempo de $0 \leq t \leq 1$. Las líneas representan cada posición del regenerador, partiendo de una temperatura de 25 grados Celsius, que corresponde a la condición inicial propuesta. La gráfica del lado derecho muestra cómo la energía va cediendo al sólido. Se estableció una condición de tipo Dirichlet en la frontera izquierda, dejando una temperatura fija de 200 grados Celsius en $x=0$. Por el lado derecho, se estableció una condición de frontera de Neumann $\frac{\partial T_g}{\partial x} = 0$ en $x=L$. Por esa razón, ese extremo va cambiando conforme se va calentando el regenerador. Los parámetros que no se incluyen por el momento solo afectan la forma de las curvas, pero el comportamiento sería similar al mostrado. 
  \newline
   En el anexo \ref{Codigo2} se muestra el código usando otra manera de resolver el modelo de Lopez-Torres usando una biblioteca de Python para resolver ecuaciones diferenciales.
  
 \newpage 
\subsection{Características del regenerador}
Se tienen los siguientes datos para un relleno de esferas de aluminio 1060, sus propiedades se describen a continuación:
\begin{table}[ht]
	\begin{center}
		\begin{tabular}{|c|c|c|}
			\hline
			\multicolumn{2}{ |c|}{Datos: Esferas} &unidades\\ \hline
			Diametro & 3 & $mm$ \\ 
			Densidad & 2705 &$kg \cdot m^{-3}$ \\
			Capacidad calorífica & 900 & $J\cdot kg^{-1} K^{-1}$ \\
			Conductividad térmica& 205 & $W \cdot m^{-1} \cdot K^{-1}$ \\
			Emisividad & 0,03& \\
			Masa por esfera & $\num{1.4e-4}$ & $kg$ \\
			
			\hline
			
			
		\end{tabular}
	\end{center}
\end{table}

\begin{table}[ht]
	\begin{center}
		\begin{tabular}{|c|c|c|}
			\hline
			\multicolumn{2}{ |c|}{Propiedades: Aire} & unidades \\ \hline
			Densidad & 1.2754  & $kg \cdot m^{-3}$ \\
			Capacidad calorífica & 1.046 &$J \cdot kg^{-1} K^{-1}$ \\
			Conductividad térmica & 0.025 & $W \cdot m^{-1} \cdot K^{-1}$ \\
			Temperatura de entrada de aire caliente& 200 & $^\circ C$ \\
			Temperatura ambiental & 30 & $^\circ C$ \\
			Viscosidad cinemática* (100$^\circ C$) &$\num{2.3e-5}$ & $m^2 \cdot s^{-1}$ \\
			Viscosidad cinemática* (200$^\circ C$) &$\num{3.45e-5}$ & $m^2 \cdot s^{-1}$ \\
			Viscosidad dinámica* (100$^\circ C$) & $\num{2.17e-5}$ & $Pa \cdot s$ \\
			Viscosidad dinámica* (200$^\circ C$) & $\num{2.57e-5}$ & $Pa \cdot s$ \\
			\hline
			
			
		\end{tabular}
	\end{center}
\end{table}
\textit{*Los valores de las viscosidades dinámica y cinemática están definidas a presión atmosférica}
\begin{table}[ht!]
	\begin{center}
		\begin{tabular}{|c|c|c|}
			\hline
			\multicolumn{2}{ |c|}{Datos del modelo:Variables} & unidades \\ \hline
			$T_b$ & Temperatura del relleno(empaquetado) & $^\circ C$ \\ 
			$T_g$ &  Temperatura del gas & $^\circ C$ \\ \hline
			\multicolumn{2}{ |c|}{Datos del modelo: Constantes} & unidades \\ \hline
			$M_b$ & Masa del relleno (empaquetado) & $kg$ \\
			$C_{p,b}$&  Capacidad calorífica del relleno & $J \cdot kg^{-1} \cdot K^{-1} $ \\
			$h_t$ & Coeficiente global de transferencia de calor & $W \cdot m^{-2} \cdot K^{-1}$ \\
			$A$ & Área de contacto & $m^2$ \\
			$m_g$ & flujo másico &  $kg \cdot s^{-1}$  \\
			$C_{p,g}$&  Capacidad calorífica del gas & $J \cdot kg^{-1} \cdot K^{-1}$\\
			$L$ & Longitud del regenerador & $m$ \\
			$M_g$ & Masa acumulada en el regenerador & $kg$ \\
			\hline
			
			
		\end{tabular}
	\end{center}
\end{table}

\begin{table}[ht]
	\begin{center}
		\begin{tabular}{|c|c|c|}
			\hline
			\multicolumn{2}{ |c|}{Datos: Regenerador} &unidades\\ \hline
			Diámetro & 12.7 & $mm$ \\ 
			Longitud & 15 &$cm$ \\
			Material  & cobre &  \\
			Capacidad calorífica & 385 & $J\cdot kg^{-1} K^{-1}$ \\
			Conductividad térmica& 385 & $W \cdot m^{-1} \cdot K^{-1}$ \\
						
			\hline
			
			
		\end{tabular}
	\end{center}
\end{table}
\newpage
\begin{figure}[ht!]
	\centering
	\includegraphics[scale=0.3]{regen_esferas.png}
	\caption{Esquema del regenerador relleno de esferas}
\end{figure}
\newpage
\subsection{Calculo de parámetros para el modelado}
\subsubsection{Propiedades del relleno del regenerador}
En este apartado, se obtienen las características necesarias para conocer la porosidad del lecho, a partir del largo del regenerador,  las dimensiones del material de relleno y la velocidad del flujo de aire de la fuente. 
 \newline
 Volumen total sin relleno:
 Se calcula el área de la tubería sin rellenar:
 \begin{equation*}
 	A_c= \frac{\pi \cdot D^2}{4} = \frac{\pi \cdot (0.0127)^2}{4} =\num{1.26676e-4} m^2
 \end{equation*}
 Obtenida el área se obtiene el volumen de la tubería:
 \begin{equation*}
 	V=A_c \cdot L = \num{1.90015e-5} m^3
 \end{equation*}
Para obtener el volumen del material de empaque ($V_p$):
\begin{equation*}
	V_p= \frac{\pi \cdot d_v^3}{6} = \num{1.41371e-8} m^3
\end{equation*}
La cantidad de esferas que puede contener:
\begin{equation*}
	n_{esferas}= \frac{V}{V_p} = 1344.083 \quad esferas
\end{equation*}
La masa de relleno del regenerador $M_b$:
\begin{equation*}
	M_{b}= n_{esferas}*\num{1.4e-4} = 0.1882 kg 
\end{equation*}
Se trunca a 1344 esferas para el volumen disponible, posteriormente obtener el volumen total del relleno:
\begin{equation*}
	V_b= n_{esferas} \cdot V = \num{1.900153e-5} m^3
\end{equation*}
La masa acumulada dentro del regenerador se obtiene de la diferencia de volumenes y el producto de la densidad del aire:
\begin{equation*}
	M_g = 0.00015 * 1.2754 = \num{1.9131e-4} kg
\end{equation*}
Para calcular la porosidad $\epsilon$:
\begin{equation*}
	\epsilon = \frac{V_b-V_p}{V_b} = 0.99925
\end{equation*}
La esfericidad es $\psi = 1$ para esferas y en caso de que no se obtiene con la siguiente formula:
\begin{equation}
	\psi = \frac{A_s}{A_p}
\end{equation}
El área de una esfera se obtiene de la siguiente fórmula:
\begin{equation*}
	A_s = 4 \pi r^2 = \num{2.82743e-5} m^2
\end{equation*}
El diámetro hidráulico de una esfera corresponde al mismo, en caso de no ser una esfera se obtiene con la siguiente relación:
\begin{equation*}
	d_h= 3mm
\end{equation*}
\begin{equation*}
	d_h= \frac{4\epsilon}{a_r(1-\epsilon)}
\end{equation*}
El parámetro de la superficie absoluta especifica, es la razón de la superficie de la partícula y el volumen del relleno:
\begin{equation*}
	a = \frac{A_s}{V_b} = 1.488 m^{-1}
\end{equation*}
\subsubsection*{Flujo másico}
El compresor seleccionado puede proporcionar 3.7585 CFM, suponiendo que tiene retoma la velocidad del flujo de la referencia\cite{Kilkovsky2020} $v=3.6 $ $m\cdot s^{-1}$, la sección transversal $S=0.0127$ y la densidad del aire $\rho_g =1.2754 $ $kg \cdot m^{-3}$
\begin{equation*}
	\dot m = \rho \cdot \ S \cdot v = (1.2754)(3.6)(0.0127) =  0.0583 kg \cdot s^{-1}
\end{equation*}
\subsection{Propiedades térmicas}
En esta sección, es de vital importancia el calculo del coeficiente total de transferencia de calor. A partir de las propiedades térmicas de las fases solida y gas del regenerador.
\newline
\subsubsection{Coeficiente global de transferencia de calor}
El coeficiente total de transferencia de calor es:
\begin{equation*}
	\frac{1}{h_t} = \frac{1}{h_{lum}} + \frac{1}{h_r} 
\end{equation*}
La difusividad térmica se puede obtener de la siguiente expresión:
\begin{equation*}
	\alpha = \frac{k}{\rho C_{p,b}} = \frac{205}{2705 \cdot 900 } = \num{8.42062e-5} m^2 s^{-1} 
\end{equation*}

Factor de Hausen para el calculo del coeficiente global de transferencia de calor, el valor de $\epsilon_H =27$ para esferas, n = 3:
\begin{equation*}
	\phi H = \frac{\pi (n+2)}{\sqrt{\epsilon_H + 18[5(\frac{n+1}{2})] }} = 1.09177
\end{equation*}
Para la siguiente expresión el diámetro de la esfera de relleno $d=0.03 m$ y la conductividad térmica del aluminio $\lambda_b =205$ $W \cdot m^{-1} \cdot K^{-1}$
\begin{equation*}
	\frac{1}{h_{lum}} = \frac{1}{h_c} + \frac{d}{2(n+2)\cdot \lambda_b} \cdot \phi H
\end{equation*}
\subsubsection{Coeficiente de transferencia de calor convectivo}
La ecuación para un relleno de un regenerador aleatorio puede obtenerse con la siguiente expresión que se obtuvo de manera experimental según la referencia\cite{AMELIO2007}, sin embargo para geometrías esféricas se puede igualar a una expresión según lo reportado por Baldwin\cite{Baldwin1966HeatTI}:
\begin{equation*}
	Nu = \frac{h_c d_p}{\lambda_g} = 0.584Re^{0.7}Pr^{1/3} 
\end{equation*}
Es necesario calcular el numero de Reynolds del regenerador producido por el relleno con la siguiente expresión:
\begin{equation*}
	Re_m = \frac{Re}{(1-\epsilon)}
\end{equation*}
De tal manera que el numero de Reynolds para el regenerador sin relleno, donde $v$ es la velocidad del flujo ($m \cdot s^{-1}$), D el diámetro hidráulico($D$) y $\nu$ la viscosidad cinemática ($m^2 \cdot s^{-1}$)\cite{Ergun1952FluidFT}:
\begin{equation*}
	Re = \frac{vD}{\nu} = \frac{3 \cdot \num{0.0127} }{\num{2.3e-5}} = 1656.5217
\end{equation*}
Por lo tanto el Reynolds para el relleno es:
\begin{equation*}
	Re_m = \frac{1656.5217}{1-0.99925} = \num{2.2087e6}
\end{equation*}
La caída de presión viene definido por la siguiente ecuación:
\begin{equation*}
	\lambda_k = \frac{150}{Re_m} +1.75 = \frac{150}{\num{2.2087e6}} + 1.75 = 1.7501
\end{equation*}
Con los datos obtenidos, se puede obtener el numero de Nusselt:
\begin{equation*}
	Nu = 0.584Re^{0.68}Pr^{1/3} = 0.584(1656.5217)^{0.68}(0.8)^{1/3} = 83.7845
\end{equation*}
El coeficiente de transferencia de calor por convención a partir del numero de Nusselt
\begin{equation*}
	h_c = \frac{Nu \cdot \lambda_g}{d_p} = \frac{83.7845 (0.025)}{0.003} = 698.2040 W\cdot m^{-1} K^{-1}
\end{equation*}
\subsubsection{Coeficiente de transferencia de calor por radiación}
La ecuación para obtener el valor de la transferencia de calor por radiación:
\newline
\textit{$\sigma = \num{5.6697e-8} $ $W \cdot m^{-2} \cdot K^{-4}$ }
\begin{equation*}
	h_r = 4\sigma \epsilon_B T_b^3 = 4(\num{5.6697e-8})(0.03)(373.15)^3 = 0.3535 W\cdot m^{-1} K^{-1} 
\end{equation*}
Por lo tanto, el coeficiente total de transferencia de calor es:
\begin{equation*}
	h_t = h_{lum} + h_r = 698.2040 + \num{1.5977e-5} + 0.3535 =  698.5575  W\cdot m^{-1} K^{-1} 
\end{equation*}
\newpage
\subsection{Resumen de parámetros}
Los parámetros obtenidos están basados según en el trabajo realizado por Kilkovsky\cite{Kilkovsky2020}, en la siguiente tabla se muestran todos los coeficientes necesarios para las ecuaciones (\ref{Gas}) y (\ref{Solido}): 
\begin{table}[ht!]
	\begin{center}
		\begin{tabular}{|c|c|c|c|}
			\hline
			\multicolumn{2}{ |c|}{Datos del modelo: Constantes}&cantidad & unidades \\ \hline
			$M_b$ & Masa del relleno (empaquetado) & 0.1882 & $kg$ \\
			$C_{p,b}$&  Capacidad calorífica del relleno & 900 & $J \cdot kg^{-1} \cdot K^{-1} $ \\
			$h_t$ & Coeficiente global de transferencia de calor&698.5575 & $W \cdot m^{-2} \cdot K^{-1}$ \\
			$A$ & Área de contacto & 0.1703 & $m^2$ \\
			$m_g$ & flujo másico &  0.0583 &$kg \cdot s^{-1}$  \\
			$C_{p,g}$&  Capacidad calorífica del gas & 1.046 &$J \cdot kg^{-1} \cdot K^{-1}$\\
			$L$ & Longitud del regenerador & 0.15 &$m$ \\
			$M_g$ & Masa acumulada en el regenerador & $\num{1.9131e-4}$ &$kg$ \\
			\hline
			
			
		\end{tabular}
	\end{center}
\end{table}
\section{Conclusión}
En conclusión, la verificación de los parámetros de un modelo matemático es una etapa crítica que garantiza la precisión y validez del mismo. A través de esta verificación, se asegura que los parámetros empleados reflejan fielmente las condiciones reales del sistema estudiado, lo cual es esencial para obtener resultados confiables.

De tal manera, la solución del modelo mediante métodos numéricos ya ha sido abordada con éxito, proporcionando una base sólida para la simulación y análisis del comportamiento del sistema. Sin embargo, este proceso no es definitivo, ya que se contempla una continua mejora y optimización de los métodos numéricos empleados. Esto permitirá una mayor precisión y eficiencia en la resolución del modelo. 


\newpage
\section{Cronograma de actividades}
\begin{figure}[ht!]
	\includegraphics[scale=0.6]{crono_grama.pdf}
\end{figure}
%%%%%%%%%%%%%%%%%%%%%%%%%%%%%%%%%%%%%%%%%%
\newpage
\bibliographystyle{ieeetr}
\bibliography{referencias_5sem}
\newpage
	\appendix
	\section{Código del método numérico}
	\label{Codigo1}
	\begin{lstlisting}[language=Python]
	import numpy as np
	import matplotlib.pyplot as plt
	
	# Parámetros
	Mb = 1.0  # Masa del sólido
	Cp_b = 1.0  # Capacidad calorífica del sólido
	mg = 1.0  # Masa del gas por unidad de longitud
	Cp_g = 1.0  # Capacidad calorífica del gas
	ht = 1  # Coeficiente de transferencia de calor
	A = 1.0  # Área de intercambio
	L = 1.0  # Longitud del dominio
	N_x = 10  # Número de puntos espaciales
	N_t = 100  # Número de puntos temporales
	dx = L / (N_x - 1)  # Paso espacial
	dt = 0.01  # Paso temporal
	
	# Inicialización de matrices
	Tb = np.zeros((N_t, N_x))
	Tg = np.zeros((N_t, N_x))
	
	# Condiciones iniciales
	Tb[0, :] = 25  # Temperatura inicial del sólido
	Tg[0, :] = 25  # Temperatura inicial del gas
	
	# Condición de frontera: temperatura constante de 200°C en el extremo izquierdo del gas
	Tg[:, 0] = 200
	
	# Iteración temporal
	for n in range(0, N_t-1):
	for i in range(1, N_x-1):
	# Ecuación (1)
	Tb[n+1, i] = Tb[n, i] + dt * ht * A * (Tg[n, i] - Tb[n, i]) / (Mb * Cp_b)
	
	# Ecuación (2)
	dTg_dx = (Tg[n, i+1] - Tg[n, i-1]) / (2 * dx)
	dTg_dt = (ht * A * (Tb[n, i] - Tg[n, i]) - mg * Cp_g * L * dTg_dx) / (Mb * Cp_g)
	Tg[n+1, i] = Tg[n, i] + dt * dTg_dt
	
	# Condición de frontera de Neumann en el extremo izquiero del solido
	Tb[:, 0] = Tb[:, 1]
	# Condición de frontera de Neumann en el extremo derecho del solido
	Tb[n+1, -1] = Tb[n+1, -2]
	# Condición de frontera de Neumann en el extremo derecho del gas
	Tg[n+1, -1] = Tg[n+1, -2]
	
	# Graficar la solución
	plt.figure(figsize=(12, 6))
	
	plt.subplot(1, 2, 1)
	for i in range(N_x):
	plt.plot(np.linspace(0, dt*N_t, N_t), Tb[:, i], label=f'Tb en x={i*dx:.2f}')
	plt.xlabel('Tiempo')
	plt.ylabel('Temperatura en el Solido')
	plt.legend()
	plt.subplot(1, 2, 2)
	for i in range(N_x):
	plt.plot(np.linspace(0, dt*N_t, N_t), Tg[:, i], '--', label=f'Tg en x={i*dx:.2f}')
	plt.xlabel('Tiempo')
	plt.ylabel('Temperatura del Gas')
	plt.legend()
	plt.savefig('solucion.png', dpi=300)
	plt.show()
\end{lstlisting}
\newpage	
\section{Ecuación homogénea de calor}
	Ejemplo 6.1:
	Considere una barra de hierro de longitud de 50 cm con un calor especifico de c = 0.437 J/g K, 
	densidad rho= 7.88 g/cm3 y conductividad térmica  k = 0.836 W/cm K. Se asume que la barra esta aislada\cite{Gockenbach2010}.
	\begin{lstlisting}[language=Python]
		'''
		Ecuacion homogenea de calor
		Ejemplo 6.1 (Mark S. Gockenbach) p.214
		Considere una barra de hierro de longitud de 50 cm con un calor especifico de c = 0.437 J/g K, 
		densidad rho= 7.88 g/cm3 y conductividad termica  k = 0.836 W/cm K. Se asume que la barra esta aislada
		'''
		from scipy.integrate import solve_ivp
		import numpy as np
		import matplotlib.pyplot as plt
		#Definir parametros
		rho=7.88
		c=0.437
		k=0.836
		#Definir las ecuciones diferenciales como funciones
		def diff_xx(u,dx,k,c,rho):
		f = (k/(rho*c))*(np.roll(u, 1)- 2*u +np.roll(u, -1))/dx
		return f
		
		#Condiciones de frontera 
		def cdf(u):
		u[0] = 0 #Condicion frontera del lado izquierdo tipo de dirichlet
		u[-1] = 0 #Condicion de frontera del lado derecho tipo dirichlet
		return u
		#Discretizacion del dominio
		N = 20 # Numero de puntos discretos
		L = 50 #Longitud de la barra de metal
		dx = L/(N-1)
		x = np.linspace(0,L,N) #Paso discreto de la longitud de la barra
		
		#Definicion de las condiciones iniciales
		u0 = 5 - (1/5)*np.abs(x-25)
		#Definicion de sistema completo de la ecuacion diferencial
		def sistema(t,Y,dx,rho,c,k): #Los argumentos de la funcion deben ir en el mismo orden
		N=len(Y)
		u=Y
		u[0] = 0 #Condicion frontera del lado izquierdo tipo de dirichlet
		u[-1] = 0 #Condicion de frontera del lado derecho tipo dirichlet
		du_dt = diff_xx(u,dx,k,c,rho)  
		return du_dt
		#Solucion de la ecuacion diferencial
		t_span = (0,10)
		sol = solve_ivp(sistema, t_span, u0, args=(dx,rho,c,k), method = 'RK45')
		print(sol)
		u_sol = sol.y # La solucion del sistema de ecuaciones diferenciales
		plt.plot(x,u_sol[:N])
		plt.xlabel('Longitud de la barra')
		plt.ylabel('Temperatura')
		plt.show()
	\end{lstlisting}	
	
	\section{Modelo del sistema de López-Torres}
	\label{Codigo2}
		\begin{lstlisting}[language=Python]
			#Datos del modelo : Constantes y variables
			import numpy as np
			T_0         =     24         #Temperatura inicial del lecho
			T_e         =     100         #Temperatura de entrada del gas
			rho_s       =    2600        #Densidad del material del lecho
			rho_g       =    1.18        #Densidad del gas
			epsilon     =    0.4         #Porosidad del lecho
			K_sx        =    1.28        #Conductividad termica efectiva del solido
			K_gx        =    0.02624     #Conductividad termica efectiva del gas
			h           =    354.4538    #Coeficiente de transferencia de calor
			a_p         =     6000       # Area superficial de las particulas por unidad de volumen
			a_w         =      20        #Area superficial interior del lecho
			u           =      0.6433    #Velocidad del flujo
			Cp_s        =     920        #Calor especifico del lecho
			Cp_g        =     1012       #Calor especifico del gas
			U           =     1.9843     #coeficente de transferencia referido al Área del lecho, entre el interior del lecho y el exterior
			# Coeficientes A, B, C, D, E
			A = 1#K_sx/(rho_s*Cp_s)
			B = 1#h*a_p/(rho_s*(1-epsilon)*Cp_s)
			C = 1#K_gx/(rho_g*Cp_g)
			D = 1#h*a_p/(rho_g*epsilon*Cp_g)
			E = 1#U*a_w/(rho_g*epsilon*Cp_g)  
			#Definicion de las funciones de las EDP
			#-------------------------Ecuacion (1)-----------------------------
			#------------------------------------------------------------------
			def diff_Ts_xx(Ts,dx):
			f = (np.roll(Ts,-1) -2*Ts + np.roll(Ts, 1) )/dx**2
			return f
			
			#-------------------------------------------------------------------
			#------------------------Ecuacion (2)-------------------------------
			def diff_Tg_xx(Tg,dx):
			f_1 = (np.roll(Tg,-1) -2*Tg + np.roll(Tg, 1) )/dx**2
			return f_1
			
			def diff_Tg_x(Tg,dx):
			f_2 = (np.roll(Tg, -1)-np.roll(Tg, 1))/(2*dx)
			return f_2
			
			#--------------------------------------------------------------------
			#--------------------Definicion de sistema completo------------------
			def sistema (t, ci, A, B, C, D, E, dx):
			N =len(ci)//2
			Ts = ci[:N]
			Tg = ci[N:]
			T0 = np.ones(N)*25
			#Se define el sistema de ecuaciones simultaneas
			dTs_dxx = diff_Ts_xx(Ts,dx)
			dTg_dx  = diff_Tg_x(Tg,dx)
			dTg_dxx = diff_Tg_xx(Tg,dx)
			dTs_dt  = A*dTs_dxx - B*(Ts - Tg)
			dTg_dt  = C*dTg_dxx + D*(Ts - Tg) - u*dTg_dx + E*(T0 - Tg)
			#Condiciones de frontera de Neumann
			dTs_dt[0]=0 #CF-izquierda 
			dTs_dt[-1]=0 #CF-derecha
			#---------------------------------
			dTg_dt[0]=0 #CF-izquierda
			dTg_dt[-1]=0 #CF-derecha
			#Condiciones de frontera de dirichlet
			Tg[0] = 200
			dT_dt = np.concatenate([dTs_dt,dTg_dt])
			return dT_dt
			#--------------------------------------------------------------------
			#-------Discretizacion del dominio-------------------------
			N = 20
			L = 1
			dx = L/(N-1)
			x = np.linspace(0,L,N)
			#Definicion de las condiciones iniciales
			Ts_0 = np.ones(N)*T_0
			Tg_0 = np.ones(N)*T_0
			#Tg_0[0] = T_e
			print(Tg_0)
			ci   = np.concatenate([Ts_0,Tg_0]) #Vector de condiciones iniciales
			#----------------Solucion del sistema de EDP
			from scipy.integrate import solve_ivp
			t_span = (0, 10)
			t_eval = np.linspace(t_span[0], t_span[1], 100)
			sol = solve_ivp(sistema, t_span, ci, args=(A, B, C, D, E, dx), t_eval=t_eval, method='RK45')
			import matplotlib.pyplot as plt
			#-------------------------Vectores de solucion----------------------
			Ts_sol = sol.y[:N, :]
			Tg_sol = sol.y[N:, :]
			plt.plot(sol.t, Tg_sol.T)
			plt.xlabel('Tiempo')
			plt.ylabel('Temperatura del gas')
			plt.show()
			plt.plot(sol.t, Ts_sol.T)
			plt.xlabel('Tiempo')
			plt.ylabel('Temperatura del solido')
			plt.show()
		\end{lstlisting}
		A continuación se presentan las graficas de solución, para el gas y para el solido después de 10 segundos.
		\begin{figure}[ht!]
			\includegraphics[scale=0.6]{temp_solido_lopez.png}
			\caption{Resultados para el solido del modelo usando Python}
		\end{figure}
		\begin{figure}[ht!]
			\includegraphics[scale=0.6]{temp_gas_lopez.png}
			\caption{Resultados para el gas del modelo usando Python}
		\end{figure}
\end{document}
